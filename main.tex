% ------------------------------------------------------------
% main.tex — Mathematical Foundations of Bioengineering
% Author: Alejandro José Soto Franco et al.
% ------------------------------------------------------------

% ------------------------------------------------------------
% preamble.tex — Mathematical Bioengineering Foundation
% Rigorous, semantically aware LaTeX preamble
% ------------------------------------------------------------

\documentclass[11pt,reqno]{amsart}

% -----------------------------
% Encoding, Font, and Language
% -----------------------------
\usepackage[utf8]{inputenc}
\usepackage[T1]{fontenc}
\usepackage[english]{babel}

% -----------------------------
% Geometry and Spacing
% -----------------------------
\usepackage[a4paper, margin=1in]{geometry}
\linespread{1.15}
\setlength{\parindent}{0pt}
\setlength{\parskip}{0.5em}

% -----------------------------
% Mathematical Environments
% -----------------------------
\usepackage{amssymb,amsmath,amsthm,mathtools}
\usepackage{stmaryrd}     % semantic brackets, etc.
\usepackage{mathrsfs}     % script fonts
\usepackage{dsfont}       % indicator functions
\usepackage{bm}           % bold math symbols

% Theorem Styles (rigorous, clean)
\newtheorem{theorem}{Theorem}[section]
\newtheorem{lemma}[theorem]{Lemma}
\newtheorem{corollary}[theorem]{Corollary}
\newtheorem{proposition}[theorem]{Proposition}
\newtheorem{axiom}[theorem]{Axiom}
\newtheorem{conjecture}[theorem]{Conjecture}

\theoremstyle{definition}
\newtheorem{definition}[theorem]{Definition}
\newtheorem{example}[theorem]{Example}
\newtheorem{notation}[theorem]{Notation}
\newtheorem{problem}[theorem]{Problem}
\newtheorem{construction}[theorem]{Construction}

\theoremstyle{remark}
\newtheorem{remark}[theorem]{Remark}
\newtheorem{observation}[theorem]{Observation}

% Proof environment
\renewcommand{\qedsymbol}{$\blacksquare$}

% -----------------------------
% Custom Mathematical Operators
% -----------------------------
\usepackage{esint}  % for \fint
\usepackage{physics} % for \dv, \grad, \curl
\usepackage{tensor}  % tensor notation
\usepackage{xparse}

\DeclareMathOperator{\Hom}{Hom}
\DeclareMathOperator{\Spec}{Spec}
\DeclareMathOperator{\im}{im}
\DeclareMathOperator{\dom}{dom}
\DeclareMathOperator{\cod}{cod}
\DeclareMathOperator{\diag}{diag}
\DeclareMathOperator{\id}{id}
\DeclareMathOperator{\supp}{supp}
\DeclareMathOperator{\vol}{vol}
\DeclareMathOperator{\Ent}{Ent}
\DeclareMathOperator{\Var}{Var}
\DeclareMathOperator{\Cov}{Cov}

\newcommand{\Set}{\mathbf{Set}}
\newcommand{\Top}{\mathbf{Top}}
\newcommand{\Man}{\mathbf{Man}}
\newcommand{\Cat}{\mathbf{Cat}}
\newcommand{\Vect}{\mathbf{Vect}}
\newcommand{\Sh}{\mathbf{Sh}}

\newcommand{\RR}{\mathbb{R}}
\newcommand{\NN}{\mathbb{N}}
\newcommand{\ZZ}{\mathbb{Z}}
\newcommand{\QQ}{\mathbb{Q}}
\newcommand{\CC}{\mathbb{C}}
\newcommand{\EE}{\mathbb{E}}
\newcommand{\PP}{\mathbb{P}}

\newcommand{\calO}{\mathcal{O}}
\newcommand{\calF}{\mathcal{F}}
\newcommand{\calL}{\mathcal{L}}
\newcommand{\calH}{\mathcal{H}}
\newcommand{\calB}{\mathcal{B}}
\newcommand{\calC}{\mathcal{C}}

\newcommand{\scrF}{\mathscr{F}}
\newcommand{\scrL}{\mathscr{L}}

% -----------------------------
% Category Theory and Diagrams
% -----------------------------
\usepackage{tikz}
\usetikzlibrary{cd,arrows.meta,decorations.markings}
\usepackage{tikz-cd} % commutative diagrams

% Nice arrows
\tikzset{
  symbol/.style={
    draw=none,
    every to/.append style={
      edge node={node [sloped, anchor=center, auto=false] {#1}}
    }
  }
}

% -----------------------------
% Cross References and Links
% -----------------------------
\usepackage{hyperref}
\hypersetup{
    colorlinks=true,
    linkcolor=blue!60!black,
    citecolor=red!50!black,
    urlcolor=purple!50!black,
    pdftitle={Mathematical Foundations of Bioengineering},
    pdfauthor={A.J. Soto Franco},
    pdfkeywords={mathematical biology, differential geometry, category theory, variational mechanics, bioengineering},
}

% Clever references
\usepackage[capitalise,noabbrev]{cleveref}
\crefformat{equation}{(#2#1#3)}

% -----------------------------
% Custom Commands for Structure
% -----------------------------
\newcommand{\todo}[1]{\textcolor{red}{TODO: #1}}
\newcommand{\note}[1]{\marginpar{\small \textcolor{blue}{#1}}}
\newcommand{\term}[1]{\textbf{#1}} % For introducing defined terms

% Typographic embellishment for emphasis
\newcommand{\highlight}[1]{\textcolor{blue!70!black}{#1}}

% Example: semantic brackets for denotation
\newcommand{\sem}[1]{\llbracket #1 \rrbracket}

% -----------------------------
% Document Metadata Macros
% -----------------------------

\usepackage{titlesec}
% ------------------------------------------------------------
% Big section headings for numbered sections only
% Quiet defaults for unnumbered sections
% ------------------------------------------------------------
\titleformat{\section}
  {\Huge\bfseries}  % formatting
  {\thesection}                % number label (e.g., "1")
  {1em}                        % spacing between number and title
  {}                           % before title (none)

% Keep unnumbered (\section*) small (defaults are fine)
\titleformat{\subsection}
  {\normalfont\large\bfseries}
  {\thesubsection}{1em}{}
\titleformat{\subsubsection}
  {\normalfont\normalsize\bfseries\itshape}
  {\thesubsubsection}{1em}{}

\usepackage{fancyhdr}
\pagestyle{fancy}

\setlength{\headsep}{20pt}  % default is ~15pt, increase as needed

% Clear default header and footer
\fancyhf{}

% Outer page numbers
\fancyfoot[C]{\vspace{1em} \\ \thepage}

% Book title on even pages (left header)
\fancyhead[LE]{\textsc{Mathematical Foundations of Bioengineering}}

% Section title on odd pages (right header)
\fancyhead[RO]{\nouppercase{\rightmark}}

\setcounter{tocdepth}{3}

\begin{document}

% ------------------------------------------------------------
% Custom Title Page
% ------------------------------------------------------------

\thispagestyle{empty}

\begin{center}
\vspace*{2.5cm}

{\fontsize{18pt}{20pt}\selectfont\scshape
Mathematical Foundations of Bioengineering\par}
\vspace{0.5cm}
{\large\itshape Geometry, Dynamics, and Symbolic Structure\par}

\vspace{1.8cm}

{\normalsize
\textsc{Alejandro Jos\'e Soto Franco}\textsuperscript{1} \\
\textsc{Meher Dhiman}\textsuperscript{2} \\
\textsc{Kevin Ji}\textsuperscript{3} \\
}

\vspace{1.5cm}
\end{center}

% Now left-aligned, smaller attribution block:
{\scriptsize
\begin{flushleft}
\textsuperscript{1}Primary and corresponding author.\\ \hspace*{1.5em}\textit{Email:} \href{mailto:sotofranco.eng@gmail.com}{sotofranco.eng@gmail.com} \\
\textsuperscript{2}Co-author: Content proof-reading and Python coding.\\
 \hspace*{1.5em}\textit{Email:} \href{mailto:meherdhiman@gmail.com}{meherdhiman@gmail.com} \\
\textsuperscript{3}Co-author: Mathematical proofs and symbolic logic.\\
 \hspace*{1.5em}\textit{Email:} \href{mailto:kevinandberbiscool@gmail.com}{kevinandberbiscool@gmail.com} \\
\end{flushleft}
}

\vspace{2cm}
\begin{center}
{\normalsize Last updated: \today\par}
\end{center}

\newpage

% ------------------------------------------------------------
% Abstract
% ------------------------------------------------------------

\begin{abstract}
This volume develops the mathematical foundations of bioengineering with full formal rigor. We treat biological systems as geometric and dynamical objects, modeled across multiple scales using the languages of differential geometry, category theory, and variational calculus. Central to our formulation is the principle that biological form and function arise from symmetry, conservation, and topological structure. We unify diverse phenomena—morphogenesis, molecular signaling, fluid mixing, and tissue mechanics—under a symbolic, computable, and semantically coherent mathematical framework.

The goal is not merely to describe but to derive: to show that life is not an exception to mathematics, but a profound expression of it.
\end{abstract}

\tableofcontents
\pagebreak
% ------------------------------------------------------------
% Introduction
% ------------------------------------------------------------

\section*{Preface}

This work is born of the conviction that the phenomena of life—its organization, motion, adaptation, and form—are fully amenable to the language of mathematics when that language is made expressive enough to meet them. Biology, in its multiscale richness, invites not simplification but expansion: an expansion of our formal vocabulary, our symbolic tools, and our geometric understanding. To meet this invitation, we build here a framework that brings the full weight of modern mathematics to bear on the structure and dynamics of living systems.

The audience for this text is broad but precise: mathematicians, physicists, and engineers who seek to construct, not merely to model; who value abstraction not for its elegance alone but for its power to illuminate and derive. We assume fluency in real analysis, linear algebra, and the differential calculus of functions and fields. Familiarity with differential geometry, measure theory, and variational principles will support a deeper engagement, but all necessary structures are developed as needed from clear foundations.

The approach taken here is integrative and axiomatic. We draw from differential geometry, category theory, symbolic logic, and continuum mechanics in the goal of pursuing a unified semantic framework. The living systems we study are represented as manifolds with structure, equipped with flows, fields, and generative rules. The mathematics is made to fit the biology via principled generalization.

At the heart of this framework lies a symbolic calculus for biological form and function. Typed, compositional, and executable, it permits biological expressions to be written and manipulated with the same clarity and rigor afforded to physical theories. From morphogen gradients to tissue dynamics, from genetic regulation to fluid transport, the language we build seeks to unify scale and structure into a coherent formalism.

Every definition in this text is chosen for its role in a broader architecture. Every theorem contributes to a semantic scaffold on which future biological theory can be constructed. We aim not to describe biology from the outside, but to inhabit its structure from within, using mathematics as the interior language of organization.

This book is the beginning of that language. It is a foundation—selective, principled, and extensible—on which more complete theories may be built. It is an invitation to the reader to participate in a new program: one that views life as structured, meaningful, and ultimately lawful in the deepest mathematical sense.

\begin{flushright}
\textit{-- A.J.S.F. \\ 27 June 2025}
\end{flushright}
\pagebreak

% ------------------------------------------------------------
% Chapters
% ------------------------------------------------------------

% ------------------------------------------------------------
% 00_intro.tex — Introduction
% ------------------------------------------------------------

\section{Introduction}

Mathematical bioengineering seeks to endow biological systems with precise mathematical representations that respect their inherent structure, scale, and dynamics. It is neither the numerical modeling of empirical data nor the heuristic construction of biological analogues; it is the formal encoding of life-like phenomena into a compositional language that admits derivation, proof, and prediction.

The aim of this text is to lay the foundations of such a language. We present a unified framework in which biological systems are expressed as structured geometric entities: manifolds equipped with flows, fibers, and symbolic grammars. The treatment draws upon modern mathematical tools while remaining grounded in concrete biological motivation.

\subsection{What is Mathematical Bioengineering?}

Mathematical bioengineering is the development of a precise, formal language for describing the structure and behavior of living systems. It is not a collection of approximations or curve-fits. It is not simulation for its own sake. It is the mathematical study of how biological form, function, and feedback emerge from geometry, dynamics, and logic.

At its core, it is the synthesis of three pillars:

\begin{itemize}
  \item \textbf{Geometric modeling}, which treats biological forms—membranes, tissues, cell aggregates—as manifolds, bundles, and embedded spaces with curvature, boundaries, and topological features.
  
  \item \textbf{Dynamical structure}, which encodes how those forms change—how they move, grow, deform, and respond—through differential equations, transport operators, and variational principles that reflect internal constraints and external forces.
  
  \item \textbf{Semantic computation}, which treats the logic of biology—gene regulation, pattern formation, control circuits—as symbolic and typed objects that can be reasoned about, transformed, and verified with mathematical precision.
\end{itemize}

This framework does not treat biology as noise around a statistical baseline. It treats biology as a generative system—structured, local, and composable. Every field, flow, and network we introduce is a well-typed mathematical object. Every model carries with it a syntax and a semantics. This allows us to move beyond vague analogies and write down the geometry and computation of life in clear mathematical terms.

\subsection{Mathematics as the Internal Language of Biology}

Biology is not chaos. It is structure, recurrence, and selection under constraint. As such, it admits a mathematics that is not imposed externally, but which arises internally from the generative mechanisms of life itself. Our thesis is that this mathematics is geometric, symbolic, and categorical in nature.

In this view, an epithelial sheet is not just a surface, but a 2-manifold embedded in space, evolving under curvature-constrained flows. A gene regulation system is not merely a network, but a diagram of typed transformations, interpreted within a functorial semantic space. A morphogen gradient is not just a scalar field, but a section of a fibered potential landscape governed by boundary conditions and transport symmetries.

The mathematical objects introduced in this text are chosen not for convenience but for their fidelity to the principles of biological construction. They obey the axioms of the systems they describe.

\subsection{On the Structure of This Text}

This work is modular and compositional. Each chapter introduces a distinct mathematical layer:
\begin{itemize}
  \item Chapter 1 defines the topological and geometric structures necessary for modeling biological space.
  \item Chapter 2 develops the theory of biological dynamics via vector fields, flows, and Lagrangian principles.
  \item Chapter 3 formalizes symmetry and conservation laws in biological systems.
  \item Chapter 4 introduces variational structures and field equations.
  \item Chapter 5 constructs stochastic extensions and measure-theoretic dynamics.
  \item Chapter 6 encodes semantic structure using type theory and category theory.
  \item Chapter 7 applies this framework to continuum models of tissue and material systems.
  \item Chapter 8 presents case studies, unifying these tools into applied biological constructions.
\end{itemize}

Each part builds upon the last, but the system is not strictly linear. The reader may consult individual sections as references, as the architecture is designed to be composable.

\subsection{Notation and Conventions}

Unless otherwise stated:
\begin{itemize}
  \item $\RR^n$ denotes $n$-dimensional Euclidean space.
  \item $M, N$ denote smooth manifolds, possibly with boundary.
  \item $\mathcal{F}, \mathscr{F}$ denote function spaces or sheaves.
  \item $\pi: E \to B$ denotes a fiber bundle with total space $E$ and base $B$.
  \item $\delta$, $\nabla$, and $\mathcal{L}_v$ denote variational, covariant, and Lie derivative operators respectively.
\end{itemize}

We will adopt geometric and variational notation where appropriate, and use type annotations when symbolic computation is involved. We now begin with the mathematical representation of biological space.
% ------------------------------------------------------------
% 01_spaces.tex — From Sets and Categories to Biologically Relevant Objects
% ------------------------------------------------------------

\newpage
\section[From Sets and Categories to Biologically Relevant Objects]{From Sets and Categories to \\ Biologically Relevant Objects}

The foundations of mathematical bioengineering must begin not with empirical models, but with the primitives of structure: sets, functions, and categories. These serve as the syntactic base upon which all subsequent notions---spaces, fields, flows, and organisms---are constructed. In this chapter, we define the minimal mathematical ontology necessary to construct the objects of biological interest.

\subsection{Sets, Functions, and Indexed Families}

We begin with the category \textbf{\textsf{Set}} of sets and functions, which serves as the foundational framework for virtually all modern mathematical formalizations \cite{maclane_categories_2010}. Objects in this category are sets, and morphisms are total functions between them. A set $X$ is understood as a collection of distinct elements, often written in roster form as $X = \{x_1, x_2, \dots\}$ or described by a defining property. A function $f : X \to Y$ is a rule that assigns to each element $x \in X$ a unique image $f(x) \in Y$; this map must be well-defined and single-valued.

The category \textsf{Set} forms the backdrop for much of elementary mathematics, but also underlies more advanced constructions in logic, algebra, and geometry. Many biological systems, although complex, can be initially abstracted to constructions over sets and functions. For example, a population of cells can be represented by a set $C$ whose elements are individual cells, and a function $s : C \to \Sigma$ may assign to each cell its current state in some state space $\Sigma$.

However, biological data is rarely homogeneous. It is frequently indexed or parameterized by additional context such as space, time, developmental stage, or anatomical compartment. To formally express such contextual variation, we make use of \textbf{indexed families}, which generalize the notion of a collection by attaching to each index a specific set, rather than grouping all data into a single undifferentiated set.

\begin{definition}[Indexed Family]
Let $I$ be a set, referred to as the index set. An \emph{$I$-indexed family of sets} is a function
\[
F : I \to \textsf{Set},
\]
which assigns to each element $i \in I$ a set $F(i)$. The collection $\{F(i)\}_{i \in I}$ represents a variable family of sets parameterized by $I$.
\end{definition}

This construction can be viewed from multiple perspectives:
\begin{itemize}
  \item As a dependent type: for each $i \in I$, there is a different type $F(i)$ of data that is appropriate at position $i$;
  \item As a fibered object: the function $F$ defines a fibration over $I$, where each fiber $F(i)$ contains the values associated with index $i$;
  \item As a functor: treating $I$ as a discrete category, $F$ is a functor from $I$ into $\textsf{Set}$, assigning to each object $i$ a set and to each morphism $\mathrm{id}_i$ the identity map on $F(i)$. 
\end{itemize}

\pagebreak

This notion is particularly relevant in biological modeling. For example:
\begin{itemize}
  \item If $I$ is a set of spatial positions within a tissue, then $F(i)$ may represent the set of molecular configurations at site $i$;
  \item If $I$ is a set of time points during development, then $F(i)$ could denote the space of phenotypic states attainable at time $i$;
  \item If $I$ indexes cell identities in a lineage tree, $F(i)$ may describe the gene expression profile or mechanical state of the corresponding cell.
\end{itemize}

Indexed families formalize the idea that data can vary along a base domain, and that this variation is not necessarily uniform. This is essential in biology, where measurements are often localized, heterogeneous, and context-dependent. The function $F$ allows the modeler to keep track of how the structure of data changes with the indexing variable, rather than collapsing all data into a single set that ignores contextual dependency.

Indexed families also serve as a bridge to more general categorical constructions. When the index set $I$ is replaced by an arbitrary small category $\mathcal{I}$, a functor $F : \mathcal{I} \to \textsf{Set}$ defines a \emph{diagram of sets}, which can capture both variation and dependency between elements of $I$. This generalization enables us to model more intricate phenomena, such as branching processes, temporal dynamics, or spatial refinement, by encoding relationships between the indexing contexts themselves.

Moreover, the concept of an indexed family is intimately tied to the notion of a \emph{bundle} in differential geometry and topology. Just as a vector bundle assigns to each point of a base space a vector space (the fiber over that point), an indexed family assigns to each index an abstract set, which may later be refined into algebraic or geometric structures. This analogy will be developed further when we consider fibered categories, sheaves, and bundles as models of biological data distributed over space or evolving over time.

From a computational perspective, indexed families correspond to dependent data types: types that are parameterized by values, common in dependently typed languages and formal verification frameworks. Such encodings are particularly important in biological computation, where one may wish to reason about properties that vary by spatial location, cell identity, or environmental context, and where the correctness of interactions depends on well-typedness across those domains \cite{pierce_types_2002}.

Thus, the concept of an indexed family plays a foundational role in building semantic models of biology that respect the compositional, local, and hierarchical nature of living systems. It provides a rigorous and flexible scaffolding upon which higher-order mathematical structures may be layered.

\subsection{Categories and Functors}

The language of category theory provides a unifying formalism in which structures, transformations, and processes can be encoded at the same level of abstraction. In contrast to set theory, which emphasizes membership, category theory emphasizes structure-preserving relationships. This is particularly aligned with the goals of mathematical bioengineering, where we seek not only to model biological data, but to model the logic and behavior of biological transformations.

\begin{definition}[Category \cite{maclane_categories_2010}]
A \emph{category} $\mathcal{C}$ consists of the following data:
\begin{itemize}
  \item A class of \emph{objects}, denoted $\mathrm{Ob}(\mathcal{C})$;
  \item For each ordered pair of objects $(A, B)$, a set of \emph{morphisms} (or arrows) $\mathrm{Hom}_{\mathcal{C}}(A, B)$ from $A$ to $B$;
  \item For each triple $(A, B, C)$ of objects, a binary operation called \emph{composition}
    \[
    \circ : \mathrm{Hom}(B, C) \times \mathrm{Hom}(A, B) \to \mathrm{Hom}(A, C),
    \]
    which assigns to each composable pair $(g, f)$ a morphism $g \circ f$;
  \item For each object $A$, a distinguished morphism $\mathrm{id}_A : A \to A$ called the \emph{identity morphism}.
\end{itemize}
This data must satisfy two axioms:
\begin{itemize}
  \item \emph{Associativity}: If $f : A \to B$, $g : B \to C$, and $h : C \to D$, then $h \circ (g \circ f) = (h \circ g) \circ f$;
  \item \emph{Identity laws}: For every morphism $f : A \to B$, we have $\mathrm{id}_B \circ f = f = f \circ \mathrm{id}_A$.
\end{itemize}
\end{definition}

A category may be seen as a minimal algebra of structure and transformation. It encodes not only the static objects of study but also the rules by which these objects relate, transform, or evolve. In biological contexts, categories may encode:
\begin{itemize}
  \item Types of spatial regions, tissues, or cell types;
  \item Morphisms representing biological processes such as differentiation, signaling, or transport;
  \item Higher-level transformations such as symmetries, feedback, or compositional rewiring.
\end{itemize}

For example, one may define a category $\mathcal{T}$ whose objects are tissue regions and whose morphisms are smooth embeddings or deformations. Alternatively, a category of developmental stages may be built whose morphisms encode time-ordered transitions or epigenetic transformations.

\begin{definition}[Functor \cite{awodey_category_2010}]
Let $\mathcal{C}$ and $\mathcal{D}$ be categories. A \emph{functor} $F : \mathcal{C} \to \mathcal{D}$ consists of:
\begin{itemize}
  \item A function on objects: for each object $A \in \mathrm{Ob}(\mathcal{C})$, an image object $F(A) \in \mathrm{Ob}(\mathcal{D})$;
  \item A function on morphisms: for each morphism $f : A \to B$ in $\mathcal{C}$, a morphism $F(f) : F(A) \to F(B)$ in $\mathcal{D}$;
\end{itemize}
such that:
\begin{itemize}
  \item $F(\mathrm{id}_A) = \mathrm{id}_{F(A)}$ for all $A \in \mathcal{C}$;
  \item $F(g \circ f) = F(g) \circ F(f)$ for all composable $f, g$.
\end{itemize}
\end{definition}

Functors preserve the compositional and identity structure of categories. They are the morphisms \emph{between} categories and thus allow us to reason about systems at a meta-level.

Functors serve several roles in mathematical bioengineering:

{\it 1. Structural interpretation.} A functor may be used to \emph{realize} an abstract biological model into a concrete computational system. For instance, a functor from a category of gene regulatory diagrams to a category of differential equations gives semantics to the abstract topology of the regulatory network.

{\it 2. Temporal evolution.} Consider a category $\mathcal{T}$ whose objects are time points and whose morphisms are time intervals. A biological system evolving over time can be modeled as a functor $F : \mathcal{T} \to \mathcal{S}$, where $\mathcal{S}$ is a category of state configurations. This describes how system states change as time progresses.

{\it 3. Spatial organization.} Let $\mathcal{O}(M)$ be the poset of open sets on a manifold $M$ (e.g., a tissue). A functor $\mathscr{F} : \mathcal{O}(M)^{\mathrm{op}} \to \textsf{Set}$ assigns data to regions of space and restriction maps to inclusions. This is the foundation of presheaf theory.

{\it 4. Experimental semantics.} One may define a functor from a category of experimental conditions to a category of observable outcomes. This formalizes how different perturbations map to system responses.

{\it 5. Modular construction.} Biological systems are highly modular. Categories and functors allow one to treat modules as objects with well-defined interfaces and to compose them via categorical operations such as colimits or pushouts.

\vspace{1em}

Just as functions are maps between sets, functors are maps between categories. And just as a function preserves the structure of its domain (via its graph or continuity, in the topological sense), a functor preserves the compositional structure of the system it represents. In this way, functors allow one to lift structure-preserving transformations from one level of abstraction to another.

In later chapters, we will encounter several categorical constructions—limits, colimits, adjunctions, and sheafifications—that depend on the interaction of categories and functors. Each of these will serve as tools for encoding, decomposing, or approximating biological phenomena in ways that are formally sound and computationally viable.


\subsection{Typed Structures and Presheaves}

Many biological observables are local in nature. That is, they depend on or vary with spatial or temporal context. For example, gene expression is often spatially restricted to a subset of a cellular population; morphogen concentrations are defined over subdomains of a tissue; curvature is computed pointwise along an embedded membrane. The mathematical formalism required to encode such locally varying data is that of \emph{presheaves}, which are functorial assignments from regions or contexts to sets of observables.

\begin{definition}[Presheaf {\cite{maclane_sheaves_1992}}]
Let $\mathcal{C}$ be a small category, often taken to be a poset $(\mathcal{U}, \subseteq)$ representing a basis for a topology or a cover of a geometric object. A \emph{presheaf} on $\mathcal{C}$ is a contravariant functor
\[
\mathscr{F} : \mathcal{C}^{\mathrm{op}} \to \textsf{Set},
\]
assigning to each object $U \in \mathcal{C}$ a set $\mathscr{F}(U)$, and to each morphism $f : V \to U$ in $\mathcal{C}$ (often an inclusion $V \subseteq U$), a \emph{restriction map}
\[
\mathscr{F}(f) : \mathscr{F}(U) \to \mathscr{F}(V),
\]
satisfying $\mathscr{F}(\mathrm{id}_U) = \mathrm{id}_{\mathscr{F}(U)}$ and $\mathscr{F}(g \circ f) = \mathscr{F}(f) \circ \mathscr{F}(g)$ for all composable morphisms $W \xrightarrow{g} V \xrightarrow{f} U$.
\end{definition}

The value $\mathscr{F}(U)$ may be interpreted as the set of sections, measurements, or states associated to the region $U$. The functoriality condition ensures that restriction to subcontexts is consistent with the categorical structure of $\mathcal{C}$. In this sense, a presheaf encodes the assignment of data to varying biological contexts in a way that respects localization and composability.

Presheaves provide a canonical formalism for expressing context-dependent types. For instance:
\begin{itemize}
  \item Let $M$ be a smooth manifold representing a biological tissue. Let $\mathcal{U}$ be an open cover of $M$, partially ordered by inclusion. A presheaf $\mathscr{F} : \mathcal{U}^{\mathrm{op}} \to \textsf{Set}$ may assign to each $U \subseteq M$ the set of morphogen profiles or gene expression vectors over $U$.
  
  \item If the values $\mathscr{F}(U)$ are not just sets but carry additional structure (e.g., vector spaces, algebras), then $\mathscr{F}$ may be defined as a functor into $\textsf{Vect}_k$ or another structured category. This permits modeling biochemical signal spaces with linear response properties.
  
  \item A time-varying tissue with changing topology may be modeled using a presheaf over a base category $\mathcal{C}$ encoding temporal slices and morphisms encoding temporal transitions or refinements of spatial data.
\end{itemize}

In general, for biological semantics, presheaves are used to encode \emph{typed observables}. A type $\tau$ may vary over space, so we define a presheaf $\mathscr{T} : \mathcal{C}^{\mathrm{op}} \to \textsf{Type}$ (or some slice category of types) so that $\mathscr{T}(U)$ specifies the types of data admissible over region $U$. The pairing of a type presheaf and a value presheaf $(\mathscr{T}, \mathscr{F})$ defines a \emph{typed fibered system}, whereby each local datum is constrained not only by position but by its semantic classification.

Presheaves can be enriched further:
\begin{itemize}
  \item A \emph{sheaf} is a presheaf that satisfies gluing and locality axioms: given a cover $\{ U_i \}_{i \in I}$ of $U$ and compatible sections $s_i \in \mathscr{F}(U_i)$ such that $s_i|_{U_i \cap U_j} = s_j|_{U_i \cap U_j}$, there exists a unique section $s \in \mathscr{F}(U)$ restricting to $s_i$ on each $U_i$.
  
  \item A \emph{prestack} generalizes this by encoding more structure into the morphism spaces (e.g., categories of sections), relevant for systems with moduli or symmetry groupoids.
\end{itemize}

In symbolic biological modeling, one may define:
\[
\text{Obs} : \mathcal{O}^{\mathrm{op}} \to \textsf{Set}
\]
where $\mathcal{O}$ is a category of biological contexts (e.g., open tissue domains, cell identities, or developmental stages), and $\text{Obs}(U)$ is the set of semantically typed observables defined over $U$.

Such constructions integrate directly with higher-level formalizations involving dependent types, fibered categories, or even sheaf cohomology when modeling topological constraints in signal propagation or morphogenetic potentials \cite{spivak_category_2014, joyal_sheaves_1993, jacobson_sheavesbio_2018}.

We will return to presheaves throughout the text, treating them as foundational structures for modeling measurable, composable, and semantically coherent biological variation across spatial and logical domains.

\subsection{From Abstract Structure to Biological Semantics}

The function of category theory in the modeling of biological systems is not merely descriptive abstraction but semantic construction. That is, we do not invoke categorical structures for elegance alone, but because they offer a formal language in which the generative and hierarchical architecture of living systems can be faithfully encoded, manipulated, and reasoned about.

Given a biological object $B$, such as a signaling system, tissue, regulatory network, or morphogenetic process, the first act of formalization is to identify the domain over which it is defined. This is often modeled as a base category or index space $I$, representing contexts such as time, spatial domain, hierarchical organization, or lineage.

\begin{itemize}
  \item If $I$ is a poset of open subsets of a tissue manifold, $B$ may be described as a presheaf of measurements or states over $I$.
  \item If $I$ is a discrete time category $\mathbf{Time}$ with objects $t_0, t_1, \dots$, then $B$ may be modeled as a functor $F : \mathbf{Time} \to \mathcal{S}$, where $\mathcal{S}$ encodes system states or phenotypic configurations.
  \item If $I$ encodes a branching developmental lineage, it may be organized into a tree-like category or groupoid over which cell-type transitions or genetic states are functorially lifted.
\end{itemize}

Once the indexing category $I$ is chosen, we assign to each object $i \in \mathrm{Ob}(I)$ a set or structured type $B(i)$ describing the local state of the system at that index. This gives rise to a functor
\[
B : I \to \mathsf{Type},
\]
or, when the system is contravariant in nature (e.g., measurement restriction, localization),
\[
B : I^{\mathrm{op}} \to \mathsf{Type},
\]
which defines the semantic content of the biological object across its domain of variation.

To model interaction, inheritance, or propagation of structure, the functoriality condition is essential: for every morphism $f : i \to j$ in $I$, there exists a map $B(f) : B(i) \to B(j)$ or $B(f) : B(j) \to B(i)$, depending on variance, capturing the transformation of biological information across scale, time, or space.

This gives rise to a collection of canonical categorical constructs recurring in biological applications:

\begin{itemize}
  \item \textbf{Signaling gradients.} Let $M$ be a smooth manifold representing a tissue, and let $\mathcal{O}(M)$ be its poset of open sets. A signaling molecule’s concentration may be modeled as a presheaf $\mathscr{F} : \mathcal{O}(M)^{\mathrm{op}} \to \mathsf{Set}$, where $\mathscr{F}(U)$ gives concentration values over $U$ and morphisms $V \subseteq U$ induce restriction maps. This structure preserves locality and enables differential operators to be defined sheaf-theoretically.
  
  \item \textbf{Feedback networks.} Consider a category $\mathcal{C}_{\text{ctrl}}$ encoding dynamical systems with feedback loops, such as control diagrams or signal flow graphs. A biological network may be modeled as a diagram $D : G \to \mathcal{C}_{\text{ctrl}}$, where $G$ is a directed graph encoding causal relations among variables or agents. Functoriality ensures composability and behavioral consistency under subsystem integration.
  
  \item \textbf{Developmental programs.} Let $\mathbf{Dev}$ be a category representing developmental stages (e.g., via time or lineage progression), and $\mathcal{S}$ a state space category (e.g., cell configurations). A developmental program is a functor $P : \mathbf{Dev} \to \mathcal{S}$ capturing the temporal unfolding of structural form. This structure supports the analysis of morphogenesis via functor cohomology or natural transformations encoding perturbations.
  
  \item \textbf{Gene regulatory architectures.} Let $\mathcal{G}$ be a quiver (directed multigraph) encoding transcription factor binding and regulation, and let $\mathcal{D}\mathcal{S}$ be a category of dynamical systems (e.g., ODEs, boolean networks). A semantic functor $F : \mathcal{G} \to \mathcal{D}\mathcal{S}$ interprets each node as a gene and each edge as a regulatory interaction with functional semantics. This allows modular interpretation, simulation, and refinement of genetic models.
\end{itemize}

The goal of this categorical modeling is not to merely encode static information, but to produce systems that support reasoning, inference, and composition. That is, given a collection of biological subsystems modeled functorially, their colimits, limits, or Kan extensions represent coherent ways of assembling, extending, or summarizing behavior across scales or contexts.

To this end, we adopt a philosophy of \emph{typed semantics}. That is, each biological observable is interpreted not merely as an element in a set, but as a typed object in a structured category. The space of types forms a semantic base, and biological objects are modeled as sections or functors over this base. The resulting semantics are compositional (via categorical limits and colimits), localizable (via sheaf and presheaf structures), and machine-verifiable (via typed symbolic encodings). By grounding biological modeling in this categorical and type-theoretic foundation, we enable a rigorous semantics for dynamic, multiscale, and spatially distributed systems. 


\subsection{Biological Objects}

Before we begin constructing formal definitions, it is worth asking: what kind of things are we actually trying to describe?

Biological objects are not arbitrary. They are structured, organized, and constrained. A single cell is not just a bag of molecules, but a system of compartments, membranes, and signaling pathways. A tissue is not just a group of cells, but a cohesive material with forces, flows, and internal symmetries. A gene network is not just a graph of interactions, but a dynamic logic, executed across time and embedded in space.

What distinguishes biological objects is their combination of structure and behavior. They are not static like geometric solids, and not purely computational like abstract machines. They are physical systems that process information, maintain form, and respond to changes in their environment. They grow, divide, repair, and differentiate. They carry memory. They enforce boundaries. They coordinate across scales.

To model these systems mathematically, we must capture more than just quantities. We must account for geometry, transformation, locality, and interaction. A “biological object” in our sense is not a single shape or variable, but a scaffold of interdependent parts, evolving according to internal rules and external influences. It might include:

\begin{itemize}
  \item A base space: the spatial or temporal domain over which the system is defined;
  \item An internal structure: what kinds of data, states, or materials inhabit each region;
  \item A dynamic rule: how the system changes, flows, or responds over time;
  \item A compositional architecture: how the object interacts with others and how it can be assembled from parts.
\end{itemize}

For example, a developing embryo is not just a collection of cells, but a continuously transforming manifold of tissue, regulated by spatial gradients, mechanical forces, and gene expression programs. A neuron is not just a node in a graph, but a polarized, electrically active geometry with discrete signaling and continuous electrochemical flow.

We begin from first principles not because biology lacks models, but because existing models are often domain-specific, inconsistent, or too empirical to explain deeper structure. Our goal is to formalize the idea that a biological object is a well-typed, semantically coherent mathematical construction: one that can be localized, composed, transformed, and analyzed with precision.

In the chapters that follow, we will make this vision precise. But it begins here—with the simple recognition that biological systems are not chaotic or incidental. They are shaped by constraints, and those constraints can be written in mathematics.



\medskip

In the next chapter, we extend this framework by introducing differential and variational structures: manifolds, vector fields, flow operators, and action principles. These tools allow us to define and analyze the dynamics of biological systems as geometric flows on structured spaces.

\newpage
\section[Structure and Dynamics in Biological Systems]{Structure and Dynamics in \\ Biological Systems}
\label{sec:structure_dynamics}

Mathematical bioengineering requires a language not just of form, but of change. Biological systems are inherently dynamic: cells divide, tissues flow, chemical concentrations vary over time and space. In this section, we develop the mathematical apparatus needed to model such phenomena on structured spaces.

\subsection{Manifolds as Biological Media}

To describe curved and continuous biological spaces, we adopt the language of smooth manifolds. This framework permits the formulation of differential geometry on topologically nontrivial spaces and enables the modeling of complex biological morphologies with mathematical rigor.

\begin{definition}[Smooth Manifold]
A smooth manifold $M$ of dimension $n$ is a second-countable, Hausdorff topological space such that for every point $p \in M$, there exists an open neighborhood $U \subseteq M$ and a homeomorphism (called a chart) $\varphi : U \to \varphi(U) \subseteq \mathbb{R}^n$, where the image is open in $\mathbb{R}^n$. The collection of such charts forms an atlas, and the manifold is said to be smooth if all transition maps $\varphi_j \circ \varphi_i^{-1}$ between overlapping charts are smooth functions.
\end{definition}

This definition provides a formal setting in which to perform differential and integral calculus, as all computations are conducted locally in coordinate space and then translated globally through smooth compatibility of charts. The maximal smooth atlas ensures maximal differentiability wherever the manifold is defined.

In biological systems, manifolds provide a natural abstraction for embedded spatial domains that exhibit curvature and connectivity but remain locally Euclidean. Examples include cellular membranes, epithelial sheets, cortical surfaces, folded protein domains, and embryonic tissues. Such structures are not flat in the ambient space but are locally diffeomorphic to open subsets of $\mathbb{R}^2$ or $\mathbb{R}^3$, allowing the differential geometry of curvature, vector fields, and flows to be meaningfully defined.

Let $f : M \to \mathbb{R}$ be a smooth scalar field defined on the manifold. In a biological context, $f$ may represent quantities such as:

\begin{itemize}
\item A morphogen concentration, which determines positional information during tissue development;
\item A temperature field, relevant in thermally mediated regulation of biochemical activity;
\item A cell density function, capturing local population levels of cells across the tissue surface.
\end{itemize}

The local behavior of $f$ at each point of $M$ provides both geometric and biological insight, and its analysis involves several differential operators that arise from the manifold's smooth structure:

\begin{itemize}
\item The \textbf{differential} $df$ is a covector field (or 1-form) defined by $df_p : T_p M \to \mathbb{R}$, assigning to each tangent vector $v \in T_p M$ the directional derivative $df_p(v) = v(f)$. This operator captures the infinitesimal rate of change of $f$ at every point and in every direction.
\item The \textbf{gradient vector field} $\nabla f$ is defined by raising the index of $df$ using the Riemannian metric $g$, such that $g(\nabla f, X) = df(X)$ for all smooth vector fields $X$. This vector field points in the direction of maximal increase of $f$ and has magnitude equal to the norm of $df$ with respect to $g$.
\item The \textbf{Laplacian} $\Delta f$ is defined as the divergence of the gradient vector field, $\Delta f = \text{div}(\nabla f)$, and it quantifies the net flux density of the gradient flow at each point. It governs diffusion-like processes across the manifold and arises in equations of the form $\partial_t f = \Delta f + R(f, x)$, where $R$ denotes a reaction term.
\end{itemize}

Each of these operators is defined intrinsically using the geometric structure of the manifold. That is, they depend only on the differential structure and the Riemannian metric, and not on any particular embedding into Euclidean space.

The use of smooth manifolds in biological modeling offers several advantages:

\begin{itemize}
\item Locality: Computations and dynamics are defined in terms of local coordinate charts, allowing models to be adapted to regions of interest.
\item Geometric invariance: The formulation is independent of coordinate choice, which is critical when studying growth, deformation, or patterning in tissues that change shape over time.
\item Compatibility with field theories: Vector fields, differential forms, and tensor fields can be naturally defined and manipulated on manifolds, supporting sophisticated models of elasticity, chemotaxis, and morphogenesis.
\end{itemize}

Furthermore, manifolds support the formal construction of additional geometric structures such as:

\begin{itemize}
\item \textbf{Tangent and cotangent bundles}, which describe local linear approximations and dual spaces;
\item \textbf{Tensor fields}, such as stress or strain fields, which encode multi-directional physical properties;
\item \textbf{Exterior differential forms}, used in modeling flux, circulation, and topological invariants;
\item \textbf{Vector bundle-valued fields}, allowing for the incorporation of multicomponent biochemical or mechanical signals at each point.
\end{itemize}

These constructions provide a foundation for describing complex biological processes using coordinate-free, geometrically faithful mathematics. In particular, the theory enables the study of spatially extended regulatory networks, curvature-mediated transport, and the deformation of active materials.

Subsequent sections will build upon this foundation to introduce curvature, flow fields, connections, and dynamical equations defined on manifolds. These tools will allow the formal modeling of biological development, signaling, and mechanics in structured and curved domains.


\subsection{Vector Fields and Flows}

Let $M$ be a smooth $n$-dimensional manifold. A \textbf{vector field} on $M$ defines a smoothly varying tangent vector at every point, representing infinitesimal displacements or directions of change.

\begin{definition}[Vector Field]
A vector field on $M$ is a smooth section $V : M \to TM$ of the tangent bundle, such that for each $p \in M$, the vector $V(p) \in T_p M$.
\end{definition}

{\it Local Expression.} In local coordinates $(x^1, \dots, x^n)$ on a chart $U \subseteq M$, a vector field $V$ has the form
\begin{equation}\label{eq:vector_field_local}
V = \sum_{i=1}^n V^i(x) \frac{\partial}{\partial x^i},
\end{equation}
where the $V^i : U \to \mathbb{R}$ are smooth scalar functions. Equation~\eqref{eq:vector_field_local} expresses $V$ in terms of a local frame of the tangent bundle, and permits coordinate-based computation of derivatives and flows.

{\it Derivation Action.} Given $f \in C^\infty(M)$, a vector field acts as a derivation on functions:
\begin{equation}\label{eq:vector_field_derivation}
V[f] := df(V) = \sum_{i=1}^n V^i(x) \frac{\partial f}{\partial x^i}.
\end{equation}
This defines a first-order linear differential operator on the algebra of smooth functions. The space of vector fields $\mathfrak{X}(M)$ can thus be viewed as a subalgebra of derivations of $C^\infty(M)$.

{\it Module Structure.} The set of all smooth vector fields on $M$ is denoted $\mathfrak{X}(M)$ and forms a module over $C^\infty(M)$. For any $f \in C^\infty(M)$ and $V \in \mathfrak{X}(M)$, the product $fV$ is a vector field given pointwise by scalar multiplication.

{\it Lie Bracket.} The Lie bracket of two vector fields $V, W \in \mathfrak{X}(M)$ is the vector field $[V, W]$ defined by
\begin{equation}\label{eq:lie_bracket}
[V, W](f) := V(W[f]) - W(V[f]) \quad \text{for all } f \in C^\infty(M).
\end{equation}
This operation is bilinear, antisymmetric, and satisfies the Jacobi identity. It endows $\mathfrak{X}(M)$ with the structure of a Lie algebra, reflecting the noncommutativity of flows.

A \textbf{flow} is a family of diffeomorphisms parameterized by time that solve an initial value problem associated with the vector field.

\begin{definition}[Flow of a Vector Field]
A \emph{flow} of $V$ is a smooth map $\phi : I \times M \to M$ for some interval $I \subseteq \mathbb{R}$ such that:
\begin{align}
\phi(0, x) &= x, \\
\label{eq:flow_ode}
\frac{d}{dt}\phi(t, x) &= V(\phi(t, x)).
\end{align}
\end{definition}

The curve $t \mapsto \phi(t, x)$ is called the \emph{integral curve} through $x$, describing the trajectory of a point under the dynamical system defined by $V$. If the flow exists for all $t \in \mathbb{R}$, it defines a one-parameter group of diffeomorphisms $\{\phi_t\}$.

{\it Biological Interpretation.} In biological systems, $V$ may represent the velocity of cellular motion, the direction of morphogen transport, the local tissue deformation field, or the flux of a biochemical species. The associated flow $\phi(t, x)$ models the evolution of a material point $x$ over time, capturing how internal or external forces shape spatial biological configurations.

{\it Riemannian Setting.} If $M$ is equipped with a Riemannian metric $g$, then vector fields can be compared using the pointwise inner product $\langle V, W \rangle_g$, and the squared norm is given by
\[
\|V\|_g^2 = g(V, V).
\]
This setting enables the definition of physically meaningful quantities such as kinetic energy density, geodesic acceleration, and curvature-driven flows. Gradient vector fields and divergence operators are defined intrinsically using the Levi-Civita connection compatible with $g$.

{\it Geometric Interpretation.} Vector fields generate infinitesimal transformations on $M$ and are the infinitesimal generators of diffeomorphism groups. Through their flows, they act on functions, tensors, and densities, giving rise to Lie derivatives. These structures form the backbone of geometric mechanics, symplectic flows, and transport theories on manifolds.

{\it Extension to Bundles.} In many applications, particularly in biological modeling, it is necessary to consider vector fields not on $M$ directly, but on a fiber bundle $\pi : E \to M$. A \emph{projectable vector field} on $E$ is one whose projection onto $M$ is a vector field $V \in \mathfrak{X}(M)$, and whose flow preserves the fibration structure. This generalization permits the modeling of systems with internal degrees of freedom—such as orientation, polarity, or chemical state—assigned to each point of the base space.

{\it Semantic Transport.} When biological structures are modeled as typed bundles or sheaves, vector fields can be interpreted as generating transformations not just in space but in semantic state. For instance, in a type-theoretic setting, the flow may update both spatial position and symbolic annotations, corresponding to dynamic biological processes like gene regulation or tissue differentiation under developmental cues.

{\it Summary.} Vector fields serve as the infinitesimal engines of motion in smooth biological systems. Their flows describe how configurations evolve, how mass or information is transported, and how complex tissue architectures deform over time. When coupled with additional structure—such as metrics, connections, or fiber semantics—they form the basis for a fully geometric and semantic theory of biological dynamics.



\subsection{Scalar Fields and Differential Operators via Jet Bundles}

Let $M$ be a smooth $n$-dimensional manifold. A scalar field on $M$ is a smooth function $f : M \to \mathbb{R}$, modeling quantities such as pressure, morphogen concentration, or metabolic density. To encode not only the function values but also their derivatives, we employ the formalism of jet bundles.

\begin{definition}[Jet Bundle of a Scalar Field]
Let $J^k(M, \mathbb{R})$ denote the $k$-jet bundle of smooth maps from $M$ to $\mathbb{R}$. A $k$-jet $j^k_p f$ at a point $p \in M$ is an equivalence class of smooth functions that agree with $f$ at $p$ up to their $k$-th order partial derivatives. The total space $J^k(M, \mathbb{R})$ is a fiber bundle over $M$, and a smooth section of $J^k(M, \mathbb{R})$ corresponds to a scalar field equipped with derivative data up to order $k$.
\end{definition}

\textit{First-Order Structure.} The first jet bundle $J^1(M, \mathbb{R})$ encodes both the scalar value and its differential. The canonical projections are:
\[
\pi_0 : J^1(M, \mathbb{R}) \to M, \qquad \pi_1 : J^1(M, \mathbb{R}) \to T^*M.
\]
Given a section $f : M \to \mathbb{R}$, the 1-jet prolongation $j^1 f : M \to J^1(M, \mathbb{R})$ maps each point $p \in M$ to its value and first derivatives.

\begin{definition}[Differential as a Jet Morphism]
The differential of $f$ is the 1-form $df : M \to T^*M$ defined at each $p \in M$ by
\begin{equation}
\label{eq:differential_def}
df_p(V) := V[f] = \left.\frac{d}{dt}\right|_{t=0} f(\gamma(t)),
\end{equation}
for any smooth curve $\gamma(t)$ with $\gamma(0) = p$, $\dot{\gamma}(0) = V \in T_p M$.
\end{definition}

\textit{Riemannian Gradient.} If $g$ is a Riemannian metric on $M$, it induces an isomorphism $g^\sharp : T^*M \to TM$ via index raising. The gradient $\nabla f$ is then the unique vector field satisfying
\begin{equation}
\label{eq:gradient_def}
g(\nabla f, V) = df(V), \qquad \forall V \in \mathfrak{X}(M).
\end{equation}
In local coordinates $(x^1, \dots, x^n)$, this yields
\begin{equation}
\label{eq:gradient_local}
\nabla f = \sum_{i,j=1}^n g^{ij} \frac{\partial f}{\partial x^j} \frac{\partial}{\partial x^i},
\end{equation}
where $g^{ij}$ is the inverse of the metric matrix $(g_{ij})$.

\textit{Divergence of Vector Fields.} Let $X \in \mathfrak{X}(M)$ be a smooth vector field. The divergence of $X$ with respect to the Riemannian volume form $\mathrm{vol}_g$ is defined as:
\begin{equation}
\label{eq:divergence}
\mathcal{L}_X \mathrm{vol}_g = (\operatorname{div} X) \cdot \mathrm{vol}_g,
\end{equation}
where $\mathcal{L}_X$ denotes the Lie derivative. In coordinates,
\begin{equation}
\label{eq:div_local}
\operatorname{div}(X) = \frac{1}{\sqrt{|g|}} \sum_{i=1}^n \frac{\partial}{\partial x^i} \left( \sqrt{|g|} X^i \right),
\end{equation}
with $|g| = \det(g_{ij})$.

\begin{definition}[Laplace--Beltrami Operator]
The Laplace--Beltrami operator acting on a smooth scalar field $f$ is defined by
\begin{equation}
\label{eq:laplace_beltrami}
\Delta f := \operatorname{div}(\nabla f).
\end{equation}
\end{definition}

\textit{Local Expression.} In local coordinates, we compute
\begin{equation}
\label{eq:laplace_beltrami_local}
\Delta f = \frac{1}{\sqrt{|g|}} \sum_{i,j=1}^n \frac{\partial}{\partial x^i} \left( \sqrt{|g|} g^{ij} \frac{\partial f}{\partial x^j} \right).
\end{equation}

\textit{Jet Interpretation.} Since $\Delta f$ involves second-order derivatives of $f$, it factors through the second jet bundle:
\[
\Delta : J^2(M, \mathbb{R}) \to \mathbb{R},
\]
and hence defines a smooth, fiberwise linear bundle morphism from $J^2(M, \mathbb{R})$ to the trivial bundle $M \times \mathbb{R}$. This makes the Laplace--Beltrami operator an intrinsic second-order differential operator on $M$.

\textit{Geometric Role.} The operator $\Delta$ governs diffusion, curvature-driven flows, and harmonic constraints in biological systems. For instance:
\begin{itemize}
  \item Solutions of $\Delta f = 0$ are harmonic functions, minimizing the Dirichlet energy
    \[
    E[f] = \int_M \|\nabla f\|^2 \, \mathrm{vol}_g.
    \]
  \item The reaction--diffusion equation
    \[
    \frac{\partial u}{\partial t} = D \Delta u + R(u)
    \]
    models spatiotemporal patterns in morphogenesis and biochemical networks.
\end{itemize}

This formalism prepares the ground for variational analysis, Hodge theory, and covariant biological PDEs in the chapters that follow.



\subsection{Reaction--Diffusion Systems on Riemannian Manifolds}

A fundamental model for spatially extended biological processes is the reaction--diffusion system. Let $(M, g)$ be a Riemannian manifold representing the spatial domain, and let $u : M \times \mathbb{R}_{\geq 0} \to \mathbb{R}$ be a scalar concentration field evolving over time. The reaction--diffusion equation takes the form
\begin{equation}
\label{eq:reaction_diffusion}
\frac{\partial u}{\partial t} = D\,\Delta u + R(u),
\end{equation}
where $\Delta$ is the Laplace--Beltrami operator associated to $g$, $D$ is a diffusion coefficient or diffusion tensor (possibly position-dependent), and $R(u)$ is a nonlinear reaction term encoding local kinetics.

Equation~\eqref{eq:reaction_diffusion} governs the interplay between diffusion—modeled as the intrinsic spread of $u$ over the curved geometry—and reaction dynamics, which may involve thresholds, saturation, or autocatalysis. This structure underlies models of morphogen gradients, Turing instabilities, and tissue-level biochemical regulation.

\subsection{Advection and Geometric Regularization of Navier--Stokes Dynamics}
\label{subsec:advection_navierstokes}

In many biological systems, scalar fields evolve not only through local chemical reactions or diffusion, but are also transported along ambient flows. This process is described by advection. Let
\[
f : M \times \mathbb{R}_{\geq 0} \to \mathbb{R}
\]
be a smooth scalar field on a Riemannian manifold $(M, g)$, representing quantities such as morphogen concentration, temperature, nutrient density, or signaling activity. Let $V \in \mathfrak{X}(M)$ be a smooth vector field representing the material flow. The evolution of $f$ under passive transport by $V$ is given by the classical \textbf{advection equation}:
\begin{equation}
\label{eq:advection_equation}
\frac{\partial f}{\partial t} + \mathcal{L}_V f = 0,
\end{equation}
where $\mathcal{L}_V f = V(f) = \langle \nabla f, V \rangle$ is the Lie derivative of $f$ along $V$. This expresses the conservation of $f$ along the integral curves of $V$. Equivalently, we write the \textbf{material derivative} condition:
\begin{equation}
\label{eq:material_derivative}
\frac{D f}{D t} := \frac{\partial f}{\partial t} + V(f) = 0,
\end{equation}
indicating that $f$ is constant when observed in the frame moving with the flow.

\subsubsection{Connection to Navier--Stokes Dynamics} In fluid mechanics, the velocity field $V$ is not given a priori, but evolves according to the \textbf{Navier--Stokes equations}. In $\mathbb{R}^3$, the incompressible Navier--Stokes system is given by:
\begin{equation}
\label{eq:navier_stokes}
\begin{cases}
\partial_t V + V \cdot \nabla V = -\nabla p + \nu \Delta V, \\
\nabla \cdot V = 0,
\end{cases}
\end{equation}
where $p$ is the pressure, $\nu > 0$ is the kinematic viscosity, and the Laplacian $\Delta V$ acts component-wise. The first equation expresses momentum conservation, while the second enforces incompressibility.

The \textbf{Navier--Stokes existence and smoothness problem}, as formulated by the Clay Mathematics Institute, asks:

\begin{quote}
Given smooth, divergence-free initial data $V_0 \in C^\infty(\mathbb{R}^3)$ with suitable decay, does there exist a unique, globally smooth solution $V(x,t)$ for all $t \geq 0$ to the Navier–Stokes equations on $\mathbb{R}^3$? Or does a finite-time singularity develop?
\end{quote}

Despite the dissipative nature of the Laplacian term, this problem remains unresolved due to the potential for nonlinear self-amplification of vorticity and singularity formation in finite time.

\textbf{Limitations of Classical Formulation.} The standard formulation of \eqref{eq:navier_stokes} assumes:
\begin{itemize}
  \item A Euclidean background geometry $(M = \mathbb{R}^3)$ with global coordinates;
  \item A strong solution framework based on classical derivatives;
  \item No intrinsic geometric, topological, or variational constraints on flow configurations.
\end{itemize}
Under these assumptions, the nonlinear advection term $V \cdot \nabla V$ can produce sharp gradients and lead to finite-time blowup of vorticity. While the Laplacian $\Delta V$ provides a dissipative smoothing effect, it may be insufficient to control the nonlinear self-interaction of the flow.

Moreover, the functional analytic setting of the problem introduces further challenges. In particular, one considers weak solutions $V \in L^\infty_t L^2_x \cap L^2_t \dot{H}^1_x$ in the sense of Leray~\cite{leray1934mouvement}, where energy is finite but regularity is weak. Such solutions are globally defined, but may not be unique, nor smooth.

The key issues include:
\begin{itemize}
  \item \textbf{Gap between weak and strong solutions:} Strong solutions in $H^s$ spaces are unique while they exist, but there is no guarantee of global regularity; weak solutions may exist globally but lack uniqueness or energy conservation.
  \item \textbf{Critical regularity:} The space $\dot{H}^{\frac{1}{2}}$ is scaling-critical for Navier–Stokes; small data well-posedness exists in some critical Besov or Morrey spaces, but large data theory remains elusive~\cite{koch2001liouville}.
  \item \textbf{Ill-posedness in borderline spaces:} Tao~\cite{tao2014ns} highlights the supercritical nature of the 3D Navier–Stokes problem and the potential nonexistence of a robust proof strategy in current PDE methods.
  \item \textbf{Lack of coercivity:} The energy inequality does not tightly control the nonlinear term without additional structure or constraints, allowing high-frequency vorticity amplification.
\end{itemize}

These limitations suggest the need for a fundamentally new approach—either in geometric structure, variational principle, or PDE formulation. The curvature-constrained flow framework we propose embeds regularity into the geometry of a bundle and may provide a new analytical route by enforcing topological and differential compatibility on the velocity field itself.


\subsubsection{Geometric Regularization via Curvature} We propose a novel regularization mechanism in which the dynamics of the velocity field $V$ are constrained not by force balance alone, but by a \textbf{curvature constraint} arising from an associated vector bundle structure.

Let $\mathcal{E} \to M$ be a smooth fiber bundle encoding internal geometric structure (e.g., vorticity, orientation, nematic order). Equip $\mathcal{E}$ with a connection $\nabla^\mathcal{E}$, whose curvature 2-form
\begin{equation}
\label{eq:curvature_form}
\Omega^\mathcal{E}(X, Y) := \left[ \nabla^\mathcal{E}_X, \nabla^\mathcal{E}_Y \right] - \nabla^\mathcal{E}_{[X,Y]}
\end{equation}
encodes the failure of horizontal distributions to close under Lie bracket. This curvature captures rotational and topological information inherent to the flow field.

We then impose a \textbf{curvature-constrained flow condition}:
\begin{equation}
\label{eq:curvature_constraint}
\iota_V \Omega^\mathcal{E} = \lambda \, d^\nabla \omega,
\end{equation}
where:
\begin{itemize}
  \item $\iota_V \Omega^\mathcal{E}$ denotes contraction of the curvature 2-form with the vector field $V$;
  \item $d^\nabla$ is the covariant exterior derivative;
  \item $\omega$ is a smooth $\mathcal{E}$-valued 1-form representing internal circulation, twist, or helicity;
  \item $\lambda$ is a coupling parameter with geometric or physical dimensions.
\end{itemize}

This relation constrains the allowed second-order structure of $V$ to satisfy a geometric compatibility with the bundle curvature. It can be viewed as a generalized constitutive relation tying vorticity generation to intrinsic geometric rotation.

\textbf{Consequences.} The regularization \eqref{eq:curvature_constraint} has several profound consequences:
\begin{itemize}
  \item It replaces the Laplacian term in \eqref{eq:navier_stokes} with a nonlinear, geometry-derived smoothing mechanism;
  \item It enforces higher regularity of $V$ through differential compatibility;
  \item It encodes topological conservation laws, such as quantized circulation or bundle holonomy invariants;
  \item It allows for singular structures (e.g., vortex lines or braids) to be smoothed not by dissipation, but by compatibility with curvature structure.
\end{itemize}

\subsubsection{Modified Advection Equation} In this setting, scalar fields such as temperature or morphogen concentration are not simply passively advected. Their evolution must respect the geometric structure of the flow. Equation \eqref{eq:advection_equation} becomes:
\begin{equation}
\label{eq:advection_curved}
\frac{\partial f}{\partial t} + \langle \nabla f, V \rangle = 0, \quad \text{subject to } \iota_V \Omega^\mathcal{E} = \lambda \, d^\nabla \omega.
\end{equation}

The advection equation is no longer autonomous—it must be solved jointly with the curvature constraint on $V$. This converts the flow problem into a coupled geometric PDE system.

\subsubsection{Jet Bundle Reformulation} In jet-theoretic language, the admissible field configurations are constrained to lie in a subbundle:
\begin{equation}
\label{eq:jet_bundle_constraint}
(j^1 f, j^2 V) \in \mathcal{C} \subset J^1(M, \mathbb{R}) \times J^2(M, TM),
\end{equation}
where $\mathcal{C}$ encodes the curvature-based constraint \eqref{eq:curvature_constraint}. This provides a precise way to define weak or distributional solutions that respect the geometry of the problem.

\textbf{Biological Applications.} In biological systems, curvature-regularized advection applies to:
\begin{itemize}
  \item \textbf{Tissue flows with embedded fiber orientation} (e.g., actomyosin networks, epithelial sheets);
  \item \textbf{Active nematic systems} with topological defect constraints;
  \item \textbf{Protein or signal transport} on deforming or non-Euclidean membranes;
  \item \textbf{Geometrically-coupled morphogen gradients} with curvature-mediated pattern localization.
\end{itemize}

By embedding fluid flow into a fiber bundle geometry and constraining its evolution via curvature compatibility, we propose a new route to resolving the singularity and existence problem posed by the classical Navier--Stokes equations. This geometric approach provides a natural language for coupling flow, structure, and transport in both physical and biological systems, potentially yielding a regular, topologically controlled theory of hydrodynamics on manifolds.



\subsection{Geometric Measures and Conservation Laws on Manifolds}

To rigorously formulate transport and conservation laws in biological systems modeled on manifolds, we must introduce a measure-theoretic framework compatible with the geometry. Scalar fields such as concentrations, densities, or morphogens must often be interpreted not merely as pointwise functions, but as quantities integrated against a volume form or transported as distributions.

\subsubsection{Riemannian Volume Measures}
\label{subsec:riemannian_volume}

Let $(M, g)$ be an oriented Riemannian manifold of dimension $n$. The metric $g$ determines a \textbf{canonical volume form}---a smooth, nowhere-vanishing differential $n$-form---denoted
\begin{equation}
\label{eq:volume_form}
dV_g = \sqrt{|g|} \, dx^1 \wedge dx^2 \wedge \cdots \wedge dx^n,
\end{equation}
where $(x^1, \dots, x^n)$ is any local coordinate chart and $|g| = \det(g_{ij})$ is the determinant of the metric tensor $g$ in those coordinates.

The wedge product
\begin{equation}
\label{eq:wedge_product}
dx^1 \wedge dx^2 \wedge \cdots \wedge dx^n
\end{equation}
is the standard generator of the space of top-degree differential forms in a coordinate patch. It is \textbf{totally antisymmetric}, multilinear, and spans the module $\Omega^n(U)$ of smooth $n$-forms over an open set $U \subseteq M$. It satisfies the alternating property
\begin{equation}
\label{eq:alternating}
dx^i \wedge dx^j = -dx^j \wedge dx^i, \quad \text{and} \quad dx^i \wedge dx^i = 0,
\end{equation}
ensuring that any $n$-form can be written uniquely as a scalar multiple of \eqref{eq:wedge_product}.

The factor $\sqrt{|g|}$ ensures that the volume form transforms properly under coordinate changes. If $\phi : U \to \tilde{U}$ is a smooth change of coordinates with Jacobian matrix $J = \partial \tilde{x}^i / \partial x^j$, then under pullback,
\begin{equation}
\label{eq:pullback_wedge}
\phi^*\left(d\tilde{x}^1 \wedge \cdots \wedge d\tilde{x}^n\right) = \det(J) \cdot dx^1 \wedge \cdots \wedge dx^n,
\end{equation}
and the Riemannian volume form transforms as
\begin{equation}
\label{eq:volume_change}
\phi^*\left( \sqrt{|\tilde{g}|} \, d\tilde{x}^1 \wedge \cdots \wedge d\tilde{x}^n \right) = \sqrt{|g|} \, dx^1 \wedge \cdots \wedge dx^n.
\end{equation}
This confirms that $dV_g$ is a \textbf{globally well-defined} top-degree form on $M$.

As a result, $dV_g$ defines a canonical Radon measure $\mu_g$ on Borel subsets of $M$, enabling coordinate-invariant integration of scalar fields:
\begin{equation}
\label{eq:volume_integral}
\int_M f \, d\mu_g := \int_M f \, dV_g.
\end{equation}

From an algebraic perspective, the space $\Omega^n(M)$ of smooth $n$-forms is a module over $C^\infty(M)$, and the volume form $dV_g$ is a generator of this module when $M$ is oriented. That is, for any smooth scalar field $f : M \to \mathbb{R}$, the product
\begin{equation}
\label{eq:form_embedding}
f \mapsto f \cdot dV_g \in \Omega^n(M)
\end{equation}
embeds $f$ as a volume density. Integration is then realized as a functional
\begin{equation}
\label{eq:integration_functional}
\int_M : \Omega^n(M) \to \mathbb{R},
\end{equation}
satisfying linearity, locality, and orientation reversal properties.

{\textbf{Aha: Coordinate Invariance via the Jacobian and Metric}}

To illustrate the power of the construction, consider a scalar field $f$ defined on $M$, and suppose a coordinate transformation $\phi : x \mapsto \tilde{x}(x)$ is applied. Naively, one might expect the integrand to pick up a factor of the Jacobian determinant:
\[
\int_M f(x) \, dx^1 \cdots dx^n \longrightarrow \int_M f(\phi^{-1}(\tilde{x})) \, \left| \det \left( \frac{\partial x}{\partial \tilde{x}} \right) \right| \, d\tilde{x}^1 \cdots d\tilde{x}^n.
\]
However, if one instead computes the integral in terms of the Riemannian volume form $dV_g$, then the metric determinant $|g|$ absorbs and cancels the Jacobian appropriately:
\[
\sqrt{|\tilde{g}(\tilde{x})|} \cdot \left| \det \left( \frac{\partial \tilde{x}}{\partial x} \right) \right| = \sqrt{|g(x)|},
\]
ensuring that
\begin{equation}
\label{eq:coordinate_invariance}
\int_M f(x) \, \sqrt{|g(x)|} \, dx^1 \cdots dx^n = \int_M f(\phi^{-1}(\tilde{x})) \, \sqrt{|\tilde{g}(\tilde{x})|} \, d\tilde{x}^1 \cdots d\tilde{x}^n.
\end{equation}
This invariance property is what guarantees that biological integrals, such as total morphogen mass or energy, remain consistent across coordinate charts---a fundamental requirement when integrating over deforming or nontrivial geometries.

{\textbf{Application in Biological Systems}}

In biological modeling, the volume form $dV_g$ allows one to integrate scalar fields such as:
\begin{itemize}
\item Cell or particle densities (e.g., $\rho(x)$),
\item Morphogen concentrations (e.g., $c(x)$),
\item Energy or signal strength distributions,
\end{itemize}
over curved tissue domains or membranes in a geometrically consistent manner. It is the essential object coupling geometry to field theory. Conservation laws, flux integrals, and variational formulations all rely on $dV_g$ to mediate between local quantities and global measurements.

This apparatus also supports extensions to manifolds with boundaries (via Stokes' theorem), to differential forms of lower degree (via the Hodge star), and to non-orientable cases (via the bundle of densities), all of which are natural next steps in a general geometric theory of biological dynamics.


\subsubsection{Continuity Equation and Reynolds Transport}
\label{subsec:continuity_reynolds}

Let $f_t : M \to \mathbb{R}$ be a time-dependent scalar field on a Riemannian manifold $(M, g)$, interpreted as a density field (e.g., mass, concentration, or energy). Let $V \in \mathfrak{X}(M)$ be a smooth vector field representing a velocity field that generates a flow $\phi_t : M \to M$. For a fixed subregion $U \subset M$, the evolution of the total quantity of $f$ over $U$ is given by the \textbf{Reynolds transport theorem}:
\begin{equation}
\label{eq:reynolds_transport}
\frac{d}{dt} \int_U f_t \, d\mu_g = \int_U \left( \frac{\partial f_t}{\partial t} + \operatorname{div}(f_t V) \right) d\mu_g.
\end{equation}

This expression accounts for both temporal change of $f$ at fixed points and spatial transport induced by the vector field $V$. When the quantity $f$ is conserved under the flow, the integrand must vanish pointwise:
\begin{equation}
\label{eq:strong_continuity}
\frac{\partial f}{\partial t} + \operatorname{div}(f V) = 0.
\end{equation}
This is the \textbf{continuity equation} in its strong form. It generalizes the classical advection equation
\[
\frac{\partial f}{\partial t} + V(f) = 0
\]
by accounting for volume distortion in the ambient geometry via the divergence operator.

\textbf{Coordinate Expression.} In local coordinates $(x^i)$ and using the Riemannian volume form $d\mu_g = \sqrt{|g|} \, dx^1 \cdots dx^n$, Equation~\eqref{eq:strong_continuity} becomes
\begin{equation}
\label{eq:local_continuity}
\frac{\partial f}{\partial t} + \frac{1}{\sqrt{|g|}} \partial_i\left( \sqrt{|g|} \, f V^i \right) = 0.
\end{equation}
This expression highlights the dependence of the flow on the geometry of the manifold. In particular, the factor $\sqrt{|g|}$ reflects local volume distortion under coordinate change.

\textbf{Integral Formulation and Weak Solutions.} Conservation laws are more generally understood in their weak (distributional) form. Let $\varphi \in C_c^\infty(M \times \mathbb{R}_{\geq 0})$ be a test function. Multiplying \eqref{eq:strong_continuity} by $\varphi$ and integrating over spacetime, one obtains:
\begin{equation}
\label{eq:weak_continuity}
\int_{\mathbb{R}_{\geq 0}} \int_M f \left( \frac{\partial \varphi}{\partial t} + \mathcal{L}_V \varphi \right) \, d\mu_g \, dt = 0,
\end{equation}
where $\mathcal{L}_V \varphi = V(\varphi)$ is the Lie derivative along $V$. This weak form supports interpretation of $f$ as a measure or distribution and is essential in singular settings (e.g., shocks, interfaces, or topological defects).

\textbf{Geometric Insight and Fluid Interpretation.} In fluid mechanics, Equation~\eqref{eq:strong_continuity} expresses the local balance of mass for an incompressible or compressible medium. If the flow $V$ is divergence-free, i.e.,
\[
\operatorname{div} V = 0,
\]
then the continuity equation simplifies to the transport equation:
\[
\frac{\partial f}{\partial t} + V(f) = 0,
\]
indicating that $f$ is passively advected by the velocity field. This case models incompressible fluids where volume is preserved along flow lines.

\textbf{Toward Geometric Regularization of Navier--Stokes.} In the context of fluid dynamics, the continuity equation forms one component of the full Navier--Stokes system. In geometric settings, one may seek regularized dynamics in which:
\begin{itemize}
  \item $f$ is a density on a bundle or sheaf over $M$;
  \item $V$ is constrained to preserve geometric or topological structures (e.g., symplectic, volume-preserving, or curvature-aligned flows);
  \item The divergence is replaced by a covariant divergence with respect to a connection on a fiber bundle;
  \item Fluxes are controlled by higher-order operators (e.g., horizontal Laplacians, sub-Riemannian diffusion).
\end{itemize}

In such regularized models, the continuity equation \eqref{eq:strong_continuity} may be modified to include bundle curvature, nontrivial connections, or defect transport. These formulations offer a pathway to resolving singularities in classical fluid models by encoding dissipation, reconnection, and topological charge conservation as intrinsic geometric invariants.


\subsubsection{Pushforward and Flow of Measures}
\label{subsec:pushforward_measures}

\begin{definition}[Flow Map]
Let $V \in \mathfrak{X}(M)$ be a smooth vector field on a manifold $M$. The associated flow map $\phi_t : M \to M$ is the solution to the initial value problem
\begin{equation}
\label{eq:flow_ode}
\frac{d}{dt} \phi_t(x) = V(\phi_t(x)), \quad \phi_0(x) = x.
\end{equation}
\end{definition}

\begin{definition}[Pushforward of a Measure]
Let $\nu_0 \in \mathcal{M}(M)$ be a Radon measure on $M$. The pushforward of $\nu_0$ under the flow map $\phi_t$ is the measure $\nu_t$ defined by
\begin{equation}
\label{eq:pushforward_def}
\nu_t := (\phi_t)_\# \nu_0.
\end{equation}
For any test function $\varphi \in C_c^\infty(M)$, the pushforward satisfies the duality relation
\begin{equation}
\label{eq:pushforward_identity}
\int_M \varphi \, d\nu_t = \int_M \varphi \circ \phi_t \, d\nu_0.
\end{equation}
\end{definition}

\begin{lemma}[Linearity and Positivity Preservation]
Let $\nu_0, \mu_0 \in \mathcal{M}^+(M)$ be Radon measures, and $\alpha, \beta \in \mathbb{R}_{\geq 0}$. Then
\begin{equation}
(\phi_t)_\#(\alpha \nu_0 + \beta \mu_0) = \alpha \, (\phi_t)_\#\nu_0 + \beta \, (\phi_t)_\#\mu_0.
\end{equation}
\end{lemma}

\begin{lemma}[Support Propagation]
Let $\mathrm{supp}(\nu_0) \subset M$ denote the support of the initial measure. Then
\begin{equation}
\mathrm{supp}(\nu_t) \subset \phi_t(\mathrm{supp}(\nu_0)).
\end{equation}
\end{lemma}

\begin{example}[Absolutely Continuous Case]
If $\nu_0 = f_0 \, d\mu_g$ for $f_0 \in L^1(M)$, then $\nu_t = f_t \, d\mu_g$ satisfies
\begin{equation}
\int_M \varphi \, f_t \, d\mu_g = \int_M \varphi \circ \phi_t \, f_0 \, d\mu_g.
\end{equation}
\end{example}

\begin{corollary}[Conservation Along Flow Lines]
For any $\varphi \in C_c^\infty(M)$ and $\nu_t = (\phi_t)_\# \nu_0$,
\begin{equation}
\frac{d}{dt} \int_M \varphi \, d\nu_t = \int_M V(\varphi) \circ \phi_t \, d\nu_0.
\end{equation}
\end{corollary}

\begin{definition}[Measure Transport Law]
The evolution of $\nu_t$ under flow satisfies the measure transport equation:
\begin{equation}
\label{eq:measure_transport_equation}
\frac{d\nu_t}{dt} + \operatorname{div}(V \nu_t) = 0
\end{equation}
in the distributional sense: for all $\varphi \in C_c^\infty(M)$,
\begin{equation}
\frac{d}{dt} \int_M \varphi \, d\nu_t = \int_M V(\varphi) \, d\nu_t.
\end{equation}
\end{definition}


\subsubsection{Weak Formulation and Measure-Valued Transport}
\label{subsec:weak_transport}

Equation~\eqref{eq:strong_continuity} admits a weak formulation suitable for distributional and measure-valued solutions. Let $f_t : M \to \mathbb{R}$ be a time-dependent scalar field on a Riemannian manifold $(M, g)$, and let $V \in \mathfrak{X}(M)$ be a smooth velocity field generating a flow $\phi_t : M \to M$. For any test function $\varphi \in C_c^\infty(M)$, the weak formulation of the continuity equation is:
\begin{equation}
\label{eq:weak_transport}
\frac{d}{dt} \int_M \varphi \, f_t \, d\mu_g = \int_M \mathcal{L}_V \varphi \cdot f_t \, d\mu_g,
\end{equation}
where $\mathcal{L}_V \varphi = V(\varphi)$ denotes the Lie derivative of the test function along $V$.

Equation~\eqref{eq:weak_transport} is obtained by integrating the strong form
\[
\frac{\partial f_t}{\partial t} + \operatorname{div}(f_t V) = 0
\]
against $\varphi$ and using integration by parts (in the sense of distributions), thereby transferring derivatives to the test function. This weak form is valid even when $f_t$ lacks differentiability, extending naturally to the setting of Radon measures and distributional densities.

In many biological applications, particularly in developmental biology or tissue mechanics, it is natural to treat $f_t$ not as a classical function but as a \emph{measure-valued field}. Let $\nu_t \in \mathcal{M}^+(M)$ denote a family of nonnegative Radon measures. The weak form of the continuity equation generalizes to:
\begin{equation}
\label{eq:measure_transport}
\frac{d}{dt} \int_M \varphi \, d\nu_t = \int_M \mathcal{L}_V \varphi \, d\nu_t,
\end{equation}
for all $\varphi \in C_c^\infty(M)$. This is the distributional transport law:
\[
\frac{d\nu_t}{dt} + \operatorname{div}(V \nu_t) = 0
\]
understood in the sense of distributions. The evolution of $\nu_t$ is entirely determined by the pushforward:
\[
\nu_t = (\phi_t)_\# \nu_0,
\]
meaning that for all test functions $\varphi$,
\[
\int_M \varphi \, d\nu_t = \int_M \varphi \circ \phi_t \, d\nu_0.
\]

This formulation is essential in modeling:
\begin{itemize}
  \item Sharp morphogen gradients or interfaces in tissue development;
  \item Localized signaling sources (e.g., organ primordia);
  \item Topological defects and singular structures in biological flows.
\end{itemize}
It enables a rigorous theory of conservation and propagation even when classical solutions fail to exist.

The measure-theoretic transport equation~\eqref{eq:measure_transport} forms the foundation for subsequent variational models and field theories over manifolds, where action functionals are expressed in terms of integrals over $d\mu_g$ and constraints are imposed via geometric compatibility or bundle curvature.


\subsubsection*{Foundations for Variational Field Theories}

The measure-theoretic formalism presented here provides the analytic and geometric foundation for variational formulations on manifolds. In subsequent sections, we will interpret biological action functionals as integrals of Lagrangian densities over $d\mu_g$, and derive field equations by applying the calculus of variations to measure-valued fields defined over jet bundles.


%\input{chapters/04_symmetry}
%\input{chapters/04_variational}
%\input{chapters/05_stochastic}
%\input{chapters/06_structures}
%\input{chapters/07_field_models}
%\input{chapters/08_case_studies}

% ------------------------------------------------------------
% Appendices
% ------------------------------------------------------------

\appendix

%\input{chapters/A_formalism}
%\input{chapters/B_background}
%\input{chapters/C_symbolic}

% ------------------------------------------------------------
% References
% ------------------------------------------------------------

\newpage
\bibliographystyle{amsalpha}
\bibliography{references}

\end{document}
