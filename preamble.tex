% ------------------------------------------------------------
% preamble.tex — Mathematical Bioengineering Foundation
% Rigorous, semantically aware LaTeX preamble
% ------------------------------------------------------------



% -----------------------------
% Encoding, Font, and Language
% -----------------------------
\usepackage[utf8]{inputenc}
\usepackage[T1]{fontenc}
\usepackage[english]{babel}

% -----------------------------
% Geometry and Spacing
% -----------------------------
\usepackage[a4paper, margin=1in]{geometry}
\linespread{1.15}
\setlength{\parindent}{0pt}
\setlength{\parskip}{0.5em}

% -----------------------------
% Mathematical Environments
% -----------------------------
\usepackage{amssymb,amsmath,amsthm,mathtools}
\usepackage{stmaryrd}     % semantic brackets, etc.
\usepackage{mathrsfs}     % script fonts
\usepackage{dsfont}       % indicator functions
\usepackage{bm}           % bold math symbols

% Theorem Styles (rigorous, clean)
\newtheorem{theorem}{Theorem}[section]
\newtheorem{lemma}[theorem]{Lemma}
\newtheorem{corollary}[theorem]{Corollary}
\newtheorem{proposition}[theorem]{Proposition}
\newtheorem{axiom}[theorem]{Axiom}
\newtheorem{conjecture}[theorem]{Conjecture}

\theoremstyle{definition}
\newtheorem{definition}[theorem]{Definition}
\newtheorem{example}[theorem]{Example}
\newtheorem{notation}[theorem]{Notation}
\newtheorem{problem}[theorem]{Problem}
\newtheorem{construction}[theorem]{Construction}

\theoremstyle{remark}
\newtheorem{remark}[theorem]{Remark}
\newtheorem{observation}[theorem]{Observation}

% Proof environment
\renewcommand{\qedsymbol}{$\blacksquare$}

% -----------------------------
% Custom Mathematical Operators
% -----------------------------
\usepackage{esint}  % for \fint
\usepackage{physics} % for \dv, \grad, \curl
\usepackage{tensor}  % tensor notation
\usepackage{xparse}

\DeclareMathOperator{\Hom}{Hom}
\DeclareMathOperator{\Spec}{Spec}
\DeclareMathOperator{\im}{im}
\DeclareMathOperator{\dom}{dom}
\DeclareMathOperator{\cod}{cod}
\DeclareMathOperator{\diag}{diag}
\DeclareMathOperator{\id}{id}
\DeclareMathOperator{\supp}{supp}
\DeclareMathOperator{\vol}{vol}
\DeclareMathOperator{\Ent}{Ent}
\DeclareMathOperator{\Var}{Var}
\DeclareMathOperator{\Cov}{Cov}

\newcommand{\Set}{\mathbf{Set}}
\newcommand{\Top}{\mathbf{Top}}
\newcommand{\Man}{\mathbf{Man}}
\newcommand{\Cat}{\mathbf{Cat}}
\newcommand{\Vect}{\mathbf{Vect}}
\newcommand{\Sh}{\mathbf{Sh}}

\newcommand{\RR}{\mathbb{R}}
\newcommand{\NN}{\mathbb{N}}
\newcommand{\ZZ}{\mathbb{Z}}
\newcommand{\QQ}{\mathbb{Q}}
\newcommand{\CC}{\mathbb{C}}
\newcommand{\EE}{\mathbb{E}}
\newcommand{\PP}{\mathbb{P}}

\newcommand{\calO}{\mathcal{O}}
\newcommand{\calF}{\mathcal{F}}
\newcommand{\calL}{\mathcal{L}}
\newcommand{\calH}{\mathcal{H}}
\newcommand{\calB}{\mathcal{B}}
\newcommand{\calC}{\mathcal{C}}

\newcommand{\scrF}{\mathscr{F}}
\newcommand{\scrL}{\mathscr{L}}

% -----------------------------
% Category Theory and Diagrams
% -----------------------------
\usepackage{tikz}
\usetikzlibrary{cd,arrows.meta,decorations.markings}
\usepackage{tikz-cd} % commutative diagrams

% Nice arrows
\tikzset{
  symbol/.style={
    draw=none,
    every to/.append style={
      edge node={node [sloped, anchor=center, auto=false] {#1}}
    }
  }
}

% -----------------------------
% Cross References and Links
% -----------------------------
\usepackage{hyperref}
\hypersetup{
    colorlinks=true,
    linkcolor=blue!60!black,
    citecolor=red!50!black,
    urlcolor=purple!50!black,
    pdftitle={Mathematical Foundations of Bioengineering},
    pdfauthor={A.J. Soto Franco},
    pdfkeywords={mathematical biology, differential geometry, category theory, variational mechanics, bioengineering},
}

% Clever references
\usepackage[capitalise,noabbrev]{cleveref}
\crefformat{equation}{(#2#1#3)}

% -----------------------------
% Custom Commands for Structure
% -----------------------------
\newcommand{\todo}[1]{\textcolor{red}{TODO: #1}}
\newcommand{\note}[1]{\marginpar{\small \textcolor{blue}{#1}}}
\newcommand{\term}[1]{\textbf{#1}} % For introducing defined terms

% Typographic embellishment for emphasis
\newcommand{\highlight}[1]{\textcolor{blue!70!black}{#1}}

% Example: semantic brackets for denotation
\newcommand{\sem}[1]{\llbracket #1 \rrbracket}

% -----------------------------
% Document Metadata Macros
% -----------------------------

\usepackage{titlesec}
% ------------------------------------------------------------
% Big section headings for numbered sections only
% Quiet defaults for unnumbered sections
% ------------------------------------------------------------
\titleformat{\section}
  {\Huge\bfseries}  % formatting
  {\thesection}                % number label (e.g., "1")
  {1em}                        % spacing between number and title
  {}                           % before title (none)

% Keep unnumbered (\section*) small (defaults are fine)
\titleformat{\subsection}
  {\normalfont\large\bfseries}
  {\thesubsection}{1em}{}
\titleformat{\subsubsection}
  {\normalfont\normalsize\bfseries\itshape}
  {\thesubsubsection}{1em}{}

\usepackage{fancyhdr}
\pagestyle{fancy}

\setlength{\headsep}{20pt}  % default is ~15pt, increase as needed

% Clear default header and footer
\fancyhf{}

% Outer page numbers
\fancyfoot[C]{\vspace{1em} \\ \thepage}

% Book title on even pages (left header)
\fancyhead[LE]{\textsc{Mathematical Foundations of Bioengineering}}

% Section title on odd pages (right header)
\fancyhead[RO]{\nouppercase{\rightmark}}

\setcounter{tocdepth}{3}